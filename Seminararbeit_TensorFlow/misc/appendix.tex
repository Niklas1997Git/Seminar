\chapter{Anhang A}
\label{chap:anhang_a}

\subsection{Code des Demonstrators}
\label{sec:codeDemonstrator}
\begin{lstlisting}
from tensorflow.examples.tutorials.mnist import input_data
mnist = input_data.read_data_sets('MNIST_data', one_hot=True)
import matplotlib.pyplot as plt
import numpy as np
import random as ran
%matplotlib inline

import tensorflow as tf
tf.reset_default_graph()

def TRAIN_SIZE(num):
    print ('Total Training Images in Dataset = ' + str(mnist.train.images.shape))
    print ('--------------------------------------------------')
    x_train = mnist.train.images[:num,:]
    print ('x_train Examples Loaded = ' + str(x_train.shape))
    y_train = mnist.train.labels[:num,:]
    print ('y_train Examples Loaded = ' + str(y_train.shape))
    print('')
    return x_train, y_train

def TEST_SIZE(num):
    print ('Total Test Examples in Dataset = ' + str(mnist.test.images.shape))
    print ('--------------------------------------------------')
    x_test = mnist.test.images[:num,:]
    print ('x_test Examples Loaded = ' + str(x_test.shape))
    y_test = mnist.test.labels[:num,:]
    print ('y_test Examples Loaded = ' + str(y_test.shape))
    return x_test, y_test

def display_digit(num):
    print(y_train[num])
    label = y_train[num].argmax(axis=0)
    image = x_train[num].reshape([28,28])
    plt.title('Example: %d  Label: %d' % (num, label))
    plt.imshow(image, cmap=plt.get_cmap('gray_r'))
    plt.show()
        
def display_mult_flat(start, stop):
    images = x_train[start].reshape([1,784])
    for i in range(start+1,stop):
        images = np.concatenate((images, x_train[i].reshape([1,784])))
    plt.imshow(images, cmap=plt.get_cmap('gray_r'))
    plt.show()
    
x_train, y_train = TRAIN_SIZE(55000)

display_digit(0)

display_mult_flat(0,400)

x = tf.placeholder(tf.float32, shape=[None, 784])
y_ = tf.placeholder(tf.float32, shape=[None, 10])
W = tf.Variable(tf.zeros([784,10]))
b = tf.Variable(tf.zeros([10]))
y = tf.nn.softmax(tf.matmul(x,W) + b)
cross_entropy = tf.reduce_mean(-tf.reduce_sum(y_ * tf.log(y), reduction_indices=[1]))

x_train, y_train = TRAIN_SIZE(5500)
x_test, y_test = TEST_SIZE(10000)
LEARNING_RATE = 0.05
TRAIN_STEPS = 1000

logs_path = '/tmp/tensorflow_logs/example/'

sess = tf.Session()
init = tf.initialize_all_variables()
sess.run(init)

training = tf.train.GradientDescentOptimizer(LEARNING_RATE).minimize(cross_entropy)
correct_prediction = tf.equal(tf.argmax(y,1), tf.argmax(y_,1))
accuracy = tf.reduce_mean(tf.cast(correct_prediction, tf.float32))

# Summary zun Darstellen von Loss
tf.summary.scalar("loss", cross_entropy)
# Summary zum Darstellen von Accuracy
tf.summary.scalar("accuracy", accuracy)
# Accuracy und Loss vereinigen
merged_summary_op = tf.summary.merge_all()

summary_writer = tf.summary.FileWriter(logs_path, graph=tf.get_default_graph())
for i in range(TRAIN_STEPS+1):
    _, c, summary = sess.run([training, cross_entropy, merged_summary_op], feed_dict={x: x_train, y_: y_train})
    summary_writer.add_summary(summary, i * TRAIN_STEPS)
    if i%100 == 0:
        print('Training Step:' + str(i) + '  Accuracy =  ' + str(sess.run(accuracy, feed_dict={x: x_test, y_: y_test})) + '  Loss = ' + str(sess.run(cross_entropy, {x: x_train, y_: y_train})))
    

for i in range(10):
    plt.subplot(2, 5, i+1)
    weight = sess.run(W)[:,i]
    plt.title(i)
    plt.imshow(weight.reshape([28,28]), cmap=plt.get_cmap('seismic'))
    frame1 = plt.gca()
    frame1.axes.get_xaxis().set_visible(False)
    frame1.axes.get_yaxis().set_visible(False) 
plt.show()


x_train, y_train = TRAIN_SIZE(1) 
display_digit(0)

answer = sess.run(y, feed_dict={x: x_train})
print(answer)

answer.argmax()

def display_compare(num):
    # THIS WILL LOAD ONE TRAINING EXAMPLE
    x_train = mnist.train.images[num,:].reshape(1,784)
    y_train = mnist.train.labels[num,:]
    
    # THIS GETS OUR LABEL AS A INTEGER
    label = y_train.argmax()
    
    # THIS GETS OUR PREDICATION AS A INTEGER
    prediction = sess.run(y, feed_dict={x: x_train}).argmax() 
    
    plt.title('Prediction: %d Label: %d' % (prediction, label))
    plt.imshow(x_train.reshape([28,28]), cmap=plt.get_cmap('gray_r'))
    plt.show()
    

display_compare(ran.randint(0, 55000))


print("Run the command line:\n" \
          "--> tensorboard --logdir=/tmp/tensorflow_logs " \
          "\nThen open localhost:6006/ into your web browser")
\end{lstlisting}
Als Grundlage für diesen Code wurde ein Beispiel des Github-Accounts \textit{aymericdamien} genutzt und kleine Änderungen vorgenommen \citep{Demonstrator}. 



\subsection{Code des Code des Beispielgraphen}
\label{sec:codeBeispielGraph}
\begin{lstlisting}
import tensorflow as tf
import numpy as np

#Initialisieren der Variablen
a = tf.Variable(3.0, name="Variable_a")
b = tf.Variable(4.0, name="Variable_b")
c = tf.Variable(1.0, name="Variable_c")
d = tf.Variable(2.0, name="Variable_d")

x_1 = tf.multiply(a, b, "Multiplikation")
x_2 = tf.add(c, d, "Addieren")
x_3 = tf.subtract(x_1, x_2, "Subtrahieren")
result = tf.sqrt(x_3, "Wurzel")

with tf.Session() as sess:
    tf.global_variables_initializer()
    sess.run(init)
    res = sess.run(result)
    print("Ergebnis der Berechnung des Graphen:", res)
    
\end{lstlisting}