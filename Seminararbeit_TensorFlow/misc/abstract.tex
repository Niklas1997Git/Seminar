%\chapter*{Überblick}
\section*{Kurzfassung}
Das Ziel der Vorliegenden Seminararbeit ist es, einen Überblick über das Thema Machine Learning mit TensorFlow zu geben und aufzuzeigen, wie ein simpler Demonstrator mithilfe dieses Frameworks programmiert werden kann. Dafür wird zuerst ein kurzer Überblick über das Thema Machine Learning im Allgemeinen gegeben, bevor auf TensorFlow im genaueren eingegangen wird. Ein neuronales Netz, welches mit TensorFlow entwickelt wurde, besteht aus einem Berechnungsgraphen. Durch diesen Graphen fließen sogenannte Tensoren. Das sind Daten, die in verschiedenen Formen auftreten können. Es können einzelne Zahlen, Datenfelder oder Matrizen sein. Diese Art der Darstellung eines neuronalen Netzes ermöglicht einen übersichtlichen Aufbau und erlaubt Parallelität, was die Berechnung beschleunigt. Als Demonstrator wurde ein neuronales Netz zum zur Erkennung von handgeschriebenen Ziffern erstellt, trainiert und ausgewertet.

\section*{Abstract}
The aim of this term paper is to summarize the topic of machine learning with TensorFlow and to give a demonstration by programming a neural network with this framework. At first, a short overview about machine learning in general is given, before TensorFlow is viewed on in detail. A neural network, which was developed with TensorFlow, consists of a computation graph. So-called tensors are flowing through this graph. Tensors are data, which can take different forms, like numbers, vectors or matrices. This representation of a neural network allows a clear structure and enables parallel computation, which increases performance. As demonstration, a neural network for handwritten digit recognition was created, trained and evaluated.
