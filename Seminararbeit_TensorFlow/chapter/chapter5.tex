\chapter{Praktisches Beispiel}
\label{chap:praktischesBeispiel}
\section{Zielsetzung}
\label{sec:zielsetzung}
\printsubchapterauthor{\authorMarco}
Zur Darstellung eines Netzes welches mithilfe von TensorFlow erstellt und trainiert wurde, soll im Zuge dieser Arbeit einen Demonstrator entwickelt werden. Das von gewählte Beispiel ist die Erkennung von handgeschriebenen Ziffern. Dabei sollen vor allem die Effekte unterschiedlicher Lernraten auf die Richtigkeit der Vorhersagen untersucht werden.

Weiterhin wird auf die Nutzung des Brechnungsgraphen eingegangen und aufgezeigt, wie dieser im Demonstrator funktioniert. Dies betrifft ebenfalls die eingesetzten Operatoren und die Transformation der Tensoren.

Im Anschluss wird auf Probleme eingegangen, welche beim Erstellen und Trainieren des Netzes aufgetreten sind.

\section{Planung}
\label{sec:planung}
\printsubchapterauthor{\authorNiklas}
1 Seite

\section{Datensätze}
\label{sec:datensaetze}
\printsubchapterauthor{\authorMarco}
Der Datensatz, den wir zum Trainieren und Überprüfen des Netzes genutzt haben, ist der MNIST Datensatz. Dieser Datensatz wurde von dem National Institute of Standards and Technology angefertigt und besteht aus 60.000 Trainingsziffern und 10.000 Testziffern. Alle Zahlen sind handgeschrieben und auf eine größe von 28x28 Pixeln gespeichert. Die einzelnen Ziffern weisen verschiedene Graustufen in ihrer Darstellung auf.

\section{Erstellen des Netzes}
\label{sec:erstellenDesNetzes}
\printsubchapterauthor{\authorNiklas}
1-2 Seiten

\section{Ergebnisse auswerten}
\label{sec:ergebnisseAuswerten}
\printsubchapterauthor{\authorMarco}

\section{Probleme}
\label{sec:probleme}
\printsubchapterauthor{\authorNiklas}
1 Seite