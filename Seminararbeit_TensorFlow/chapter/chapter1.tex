\chapter{Einleitung}  
\label{chap:einleitung}
\chapterauthor{\authorMarco}

\section{Motivation}
\label{sec:motivation}
Das Thema Machine Learning tritt in der heutigen Zeit immer öfter im alltäglichen Leben auf. Es gibt digitale Sprachassistenten, Produktvorschläge oder autonom fahrende Autos. Hinter all dem verbirgt sich Machine Learning. Doch Machine Learning ist nicht nur ein Thema, mit dem sich große Firmen beschäftigen können. Auch privat kann jeder seine eigenen Machine Learning Projekte durchführen und für sich selbst ein sinnvolles Programm erstellen. Dafür muss nicht alles von Neuem erfunden werden. Es gibt bereits viele verschiedene Frameworks, welche beim Erstellen, Trainieren und Benutzen von Modellen mithilfe von Machine Learning helfen. Eines dieser Frameworks ist TensorFlow.

\section{Zielsetzung}
\label{sec:zielsetzung}
Im Laufe dieser Arbeit soll ein allgemeiner Überblick über das Thema Machine Learning gegeben und die Funktionsweise von TensorFlow erläutert werden. Dafür wird zusätzlich zu den theoretischen Grundlagen ein Demonstrator gezeigt, welcher ein grundlegendes neuronales Netz mithilfe von TensorFlow trainiert.

\section{Aufbau}
\label{sec:aufbauArbeit}
Im ersten Kapitel wird generell auf Machine Learning eingegangen. Dabei werden Ziele erläutert und verschiedene Möglichkeiten des Trainierens eines Netzes aufgezeigt. Weiterhin werden Anwendungsmöglichkeiten von Machine Learning beschrieben. Anschließend wird das Framework TensorFlow vorgestellt und dessen Aufbau und Arbeitsweise erklärt. Danach wird der Demonstrator vorgestellt und auf die Erstellung sowie aufgetretene Probleme eingegangen. Abschließend werden noch einmal Vor- und Nachteile des Frameworks diskutiert und ein Ausblick auf die Zukunft von TensorFlow gegeben.