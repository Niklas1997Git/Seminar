\chapter{TensorFlow}
\label{chap:tensorflow}
\chapterauthor{\authorNiklas}

Worum geht es in Ihrer Arbeit \citep{Einfuehrung}
Insgesamt 2 Seiten

\section{Allgemeines}
\label{sec:allgemeines}
"Die Daten fließen als Tensoren durch einen Graphen von Rechenoperationen, aus denen unser Deep-Learning-Netz besteht." \cite{Einfuehrung}\\
Berechnung als Datenflussgraph\\
Knoten Darstellung von Operationen\\
Kanten Daten (Tensoren)\\
Berechnungen mit Datenflussgraphen\\
für maschinelles Lernen entwickelt\\
Portierbarkeit (unterschiedliche Hardware)\\
Hauptkomponenten in C++\\
Python am weitesten entwickelt und am meisten genutzt\\
Flexibilität, Modelle sind einfach zu erstellen\\
viele Optimierungsverfahren, der Nutzer wird bei Verwendung dieser unterstützt\\
Prozess kann durch Tensorboard beobachtet werden\\
Abstraktionsbibliotheken, wie Keras vereinfachen Benutzung
Modelle können schneller in den Betrieb gehen\\
Modell für einzelnen Prozessor kann durch wenige Änderungen
auf Prozessorclustern laufen\\
Spezielle Tensor Processing Units von Google\\
Wird ständig verbessert in den Bereichen Benutzerfreundlichkeit, Leistungsfähigkeit und Nützlichkeit
Hier wird auf die Textstelle \ref{sec:tensoren} verwiesen, die
sich auf der Seite \pageref{sec:tensoren} befindet.

\section{Unterschiede zu anderen Frameworks}
\label{sec:unterschiede}
