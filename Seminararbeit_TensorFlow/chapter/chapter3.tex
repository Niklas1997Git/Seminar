\chapter{TensorFlow}
\label{chap:tensorflow}
%\chapterauthor{\authorNiklas}
\section{Allgemeines}
\label{sec:allgemeines}
\printsubchapterauthor{\authorNiklas}
TensorFlow ist ein Framework, das 2015 von Google veröffentlicht wurde, um das Erstellen und Arbeiten mit neuronalen Netzen zu vereinfachen. Dieses Framework bearbeitet mathematische Berechnungen auf Basis eines Datenflussgraphen."Die Daten fließen als Tensoren durch einen Graphen von Rechenoperationen" \citep{Einfuehrung}. Daher kommt auch der Name TensorFlow. Obwohl TensorFlow für grundlegende mathematische Berechnungen genutzt werden kann, ist die Hauptfunktion das Unterstützen des maschinellen Lernens. TensorFlow bietet eine große Portierbarkeit. Mit nur kleinen Änderungen am Code kann dieser auf unterschiedlichen Geräte, einschließlich Mobilgeräten, eingesetzt werden. Nutzer dieses Frameworks genießen ebenfalls eine hohe Flexibilität, da Modelle von neuronalen Netzen einfach erstellt werden können und viele Abstraktionbibliotheken, wie Keras, die Bentzung stark vereinfachen. Das sogenannte Tensorboard unterstützt den Nutzer bei der Suche von Fehlern und visualisiert den Datenflussgraphen. \cite{Einfuehrung} 

Die Hauptkomponenten wurden in C++ entwickelt. Die am weitesten entwickelte und am häufigsten genutzte Schnittstelle ist die Python-Schnittstelle. TensorFlow wird ständig von Google weiterentwickelt, um die Leistungsfähigkeit, die Benutzerfreundlichkeit und die Nützlichkeit zu steigern. Dazu gehören zum Beispiel auch die \textit{Tensor Processing Units}, auf die in der folgenden Arbeit noch weiter eingegangen wird.

%\section{Unterschiede zu anderen Frameworks}
%\label{sec:unterschiede}
%\printsubchapterauthor{\authorNiklas}
TensorFlow besitzt gegenüber anderen Frameworks den Vorteil, das neuronale Netzwerk in kleinere Abschnitte zu unterteilen. Dies ist möglich da TensorFlow mit einem Datenflussgraphen arbeitet. Eine weitere Steigerung der Rechenleistung bietet Google durch spezielle Recheneinheiten, die nur für TensorFlow entwickelt wurde. Die sogenannten \textit{TensorFlow Processing Units} Außerdem besitzt TensorFlow die Möglichkeit seine Graphen zu visualisieren und dadurch die Fehlersuche deutlich zu verbessern. Diese Visualisierung nennt sich Tensorboard. Die graphische Darstellung wird in dem Kapitel \ref{sec:graphenAufbau} beschrieben. Ein weiterer Unterschied zu anderen Framework besteht in der Portierbarkeit. TensorFlow kann auf sehr vielen Systemen genutzt werden, was mit anderen Frameworks nicht möglich ist.
